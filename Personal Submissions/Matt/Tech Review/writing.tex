\documentclass[onecolumn, draftclsnofoot,10pt, compsoc]{IEEEtran}
\usepackage{graphicx}
\usepackage{url}
\usepackage{setspace}
\bibliographystyle{IEEEtran}

\usepackage{geometry}
\geometry{textheight=9.5in, textwidth=7in}

% 1. Fill in these details
\def \CapstoneTeamName{		TaalSquad}
\def \CapstoneTeamNumber{		33}
\def \GroupMemberOne{			Matthew Ruder}
\def \CapstoneProjectName{		Tracking radiotagged animals with automated drones and VHF antennas}
\def \CapstoneSponsorCompany{	Levi Lab, Department of Fisheries and Wildlife, Oregon State University}
\def \CapstoneSponsorPerson{		Taal Levi}

% 2. Uncomment the appropriate line below so that the document type works
\def \DocType{		Tech Review
				%Requirements Document
				%Technology Review
				%Design Document
				%Progress Report
				}
			
\newcommand{\NameSigPair}[1]{\par
\makebox[2.75in][r]{#1} \hfil 	\makebox[3.25in]{\makebox[2.25in]{\hrulefill} \hfill		\makebox[.75in]{\hrulefill}}
\par\vspace{-12pt} \textit{\tiny\noindent
\makebox[2.75in]{} \hfil		\makebox[3.25in]{\makebox[2.25in][r]{Signature} \hfill	\makebox[.75in][r]{Date}}}}
% 3. If the document is not to be signed, uncomment the RENEWcommand below
%\renewcommand{\NameSigPair}[1]{#1}

%%%%%%%%%%%%%%%%%%%%%%%%%%%%%%%%%%%%%%%
\begin{document}
\begin{titlepage}
    \pagenumbering{gobble}
    \begin{singlespace}
    	\includegraphics[height=4cm]{coe_v_spot1}
        \hfill 
        % 4. If you have a logo, use this includegraphics command to put it on the coversheet.
        %\includegraphics[height=4cm]{CompanyLogo}   
        \par\vspace{.2in}
        \centering
        \scshape{
            \huge CS Capstone \DocType \par
            {\large\today}\par
            \vspace{.5in}
            \textbf{\Huge\CapstoneProjectName}\par
            \vfill
            {\large Prepared for}\par
            \Huge \CapstoneSponsorCompany\par
            \vspace{5pt}
            {\Large\NameSigPair{\CapstoneSponsorPerson}\par}
            {\large Prepared by }\par
            Group\CapstoneTeamNumber\par
            % 5. comment out the line below this one if you do not wish to name your team
            \CapstoneTeamName\par 
            \vspace{5pt}
            {\Large
                \NameSigPair{\GroupMemberOne}\par
            }
            \vspace{20pt}
        }
        \begin{abstract}
        % 6. Fill in your abstract    
        	This document serves to divide the core issue with current methods of animal tracking into individual system components. The goal of this project is to reduce the cost of tracking forest animals and obtain reliable data. To address this issue, drones were declared a viable option if mounted with VHF antennas. With proper design and programming, the drones would be able to successfully locate tagged creatures, essentially reducing the amount of time, cost, and effort into tracking these animals. By addressing the individual components of the project, drones shall be constructed to automate the tracking of these animals. 
        \end{abstract}     
    \end{singlespace}
\end{titlepage}
\newpage
\pagenumbering{arabic}
\tableofcontents
% 7. uncomment this (if applicable). Consider adding a page break.
%\listoffigures
%\listoftables
\clearpage

% 8. now you write!
\section{Introduction}
Tracking large animals is straightforward within open environments because of GPS tracking, which allows for pinpoint measurements from anywhere in the forest; however, the footprint of GPS transmitters along with the associated battery pack is too cumbersome for smaller animals, such as the Marbled Murrelet which the Department of Fisheries and Wildlife desperately wants to track. Current methods of tracking forest creatures involves utilizing VHF radio technology. Selected animals are given radio tags, which emit a unique radio frequency that is picked up by a VHF antennas. VHF antennas require manual assistance in order to successfully track creatures; as such, the cost of tracking animals using VHF antennas is extremely expensive, requires manual assistance, and systematically inefficient. The ultimate goal of the project is to reduce the cost of tracking forest animals while obtaining reliable data. In order to accomplish this, several steps must be taken to construct a proper solution. The main objective is finding a VHF signal. The device designed must be able to receive VHF signals within densely forested areas. If the device cannot accomplish this task, then it does not solve the problem. Secondly, the device must be automated and capable of operating within a dense forest environment; as such, battery life must be taken into account. The constructed device must be able to last several hours in order to cover the largest area possible. The frame of the drone must also be taken into account, considering it must operate in a densely forested environment; therefore, it is crucial that the drone can properly takeoff and land in this environment. Finally, translating the data of the flight is necessary for the drone to be a workable solution to the problem. Potential software to process this data shall be explored.

\section{Tech Piece 1: VTOL}
As stated within the introduction, tracking creatures within densely forested areas is costly and inefficient. To fully automate the process of tracking creatures, drones equipped with VHF antennas will enable wildlife researchers indicators as to where tagged creatures are currently located. One of major issues with designing the drone is the type of frame to be used. Multi-rotor and fixed wing frames are possible solutions; however, coverage and landing requirements are problematic to the solutions respectively. To address these issues, a vertical take-off and landing (VTOL) frames shall be explored throughout this section.
\subsection{VTOL Option 1}
One possible VTOL frame to use is the FunCub QuadPlane, which operates as a standard tailplane aircraft. The frame requires 900 kV motors. Four vertical propellers are used during takeoff before transitioning to a standard plane configuration during flight. The frame lands in a similar manner, where when the frontal motors slow down, the vertical propellers take control of the landing procedure. This frame can hold up to 2.3 kg of payload.[1]
\subsection{VTOL Option 2}
A potential, and expensive, frame that can be used is the New Mugin 2930mm H-Tail VTOL package, complete with necessary construction components and 44v motors. Similar to the FunCub QuadPlane, the frame takeoff uses four vertical propellers before transitioning to a horizontal motor. The main difference between this frame and the first option is the increased size and payload of the frame, enabling a carrying capacity of up to 8 kg.[2]
\subsection{VTOL Option 3}
Another possible VTOL frame is the TBS Caipiroshka frame. The frame is a V-shape with two 1800 kV motors on each wing. Unlike the previous two frames, this one is set vertically such that the nose is pointed to the sky. During takeoff, the frame flies upwards before accelerating into flight mode. Due to the increased weight near the back of each wing, the frame is able to return to vertical position when slowing down for a landing. One major problem with this frame is that the weight distribution needs to be correct in order to takeoff and land correctly.[3]
\section{Tech Piece 2: Batteries}
One of the main issues with drones is battery life. The lifeline and length of a drone's flight heavily relies on its energy consumption. To mitigate the issue of energy consumption, the total energy available to the drone must be high enough to perform its task of tracking animals within 100 square kilometers. Taking the type of frame used into consideration, the capacitance and load of the battery will be dependent on the total weight of the drone's payload. As such, possible battery options will be listed in this section.

\subsection{Batteries Option 1}
One possible battery is the MaxAmps MA6s11000 Lithium Polymer cell array, consisting of 22.2 V, 11000 mAh capacity, and 1.2 kg mass. Using this relatively expensive battery, a drone can fly for about an hour. This series of battery is fit for drones with lighter payloads such as hexacopters.[4]
\subsection{Batteries Option 2}
Another possible battery is the MaxAmps MA6s10900 Lithium Polymer cell array, consisting of 22.2 V, 10900 mAh capacity, and 1.3 kg mass. Similar to the first option, this battery can power a drone for about an hour. This series of battery is fit for drones with heavier payloads such as octocopters.[4]
\subsection{Batteries Option 3}
A cheaper alternative to the two batteries would be the Panasonic NCR 18650B Lithium Ion cell, with 3.6 V, 3200 mAh capacity, and 47g mass. Unlike the previous two options, this is a single cell and does not hold much charge. To compensate, it is incredibly cheap and can be spot welded together to form a cell array. About 21 of these cells can be combined to output the same amount of power as the first option. The largest downside to this method is the increased possibility of fire hazards due to short circuits or improper welding.[4]
\section{Tech Piece 3: Tools and Languages}
During the flight of the drone, the VHF antennas will be collecting and storing the frequency of signals received over time. Post-flight software will need to be utilized in order to translate collected data points into comprehensible and useful positional information. In addition, the coding languages utilized alongside these tools shall be investigated within this section.
\subsection{Tools and Languages Option 1}
One of the possible data processing modules to use is MATLAB, where the RTL-SDR receiver acquires Aeronatical VDL2 data link messages and are processed using MATLAB. For example, designfilt() and filtfit() functions can be used to filter out ambient noise within the data set. In essence of technicality, MATLAB can be used to filter, demodulate, and time-sync data acquired by the VHF receiver.[5]
\subsection{Tools and Languages Option 2}
Python as a possible method for RTL-SDR data processing. In addition, the response offers heavy insight into VHF signal detection, listing instructions on how to optimize data collection quality and performance.[6]
\subsection{Tools and Languages Option 3}
Airspy offers a pre-built software-defined radio package known as SDR\# or SDRSharp, which features built-in signal analysis and demodulation. The software is capable of listening to VHF signals and is compatible with any RTL-SDR receivers.[7]

\section{Conclusion}
The proposed completion of the project is when a working drone is delivered with capability of recording GPS coordinates and direction of strongest signal relative to itself. By selecting the proper frame, battery, and data translation software, the project will be several steps closer to completion. If the project solution can be used as a replacement to the current manual process of tracking animals, then the project will be considered a success. Ideally, the system should be able to handle data in densely forested areas. The end result of this project must result in a working prototype or multiple prototypes that can cover at least 100 square kilometers of radio tracking. The drones must be able to fly above the tree line of mountainous forested areas without crashing. The agreed upon cost of each drone should fall within the \$6,000 replication cost. The components and blueprints of each drone must be listed and printed such that the drones may be reproduced in the future. Programming and code of each drone must be available to the client with instructions on how to program the drones. The prototype should be able to collect data without manual intervention or assistance. The drones must locate animals faster than manual tracking. The data collected must accurately locate an animal within the forest. 

\nocite{*}
\bibliography{references}
\end{document}