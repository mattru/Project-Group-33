\documentclass[onecolumn, draftclsnofoot,10pt, compsoc]{IEEEtran}
\usepackage{graphicx}
\usepackage{url}
\usepackage{setspace}

\usepackage{geometry}
\geometry{textheight=9.5in, textwidth=7in}

% 1. Fill in these details
\def \CapstoneTeamName{		The Cleverly Named Team}
\def \CapstoneTeamNumber{		777}
\def \GroupMemberOne{			David Hume}
\def \GroupMemberTwo{			Thomas Kuhn}
\def \GroupMemberThree{			Karl Popper}
\def \CapstoneProjectName{		Building a Robot Using Only Philosophy}
\def \CapstoneSponsorCompany{	Cheap Robots, Inc}
\def \CapstoneSponsorPerson{		Roger Bacon}

% 2. Uncomment the appropriate line below so that the document type works
\def \DocType{		Problem Statement
				%Requirements Document
				%Technology Review
				%Design Document
				%Progress Report
				}
			
\newcommand{\NameSigPair}[1]{\par
\makebox[2.75in][r]{#1} \hfil 	\makebox[3.25in]{\makebox[2.25in]{\hrulefill} \hfill		\makebox[.75in]{\hrulefill}}
\par\vspace{-12pt} \textit{\tiny\noindent
\makebox[2.75in]{} \hfil		\makebox[3.25in]{\makebox[2.25in][r]{Signature} \hfill	\makebox[.75in][r]{Date}}}}
% 3. If the document is not to be signed, uncomment the RENEWcommand below
%\renewcommand{\NameSigPair}[1]{#1}

%%%%%%%%%%%%%%%%%%%%%%%%%%%%%%%%%%%%%%%
\begin{document}
\begin{titlepage}
    \pagenumbering{gobble}
    \begin{singlespace}
    	\includegraphics[height=4cm]{coe_v_spot1}
        \hfill 
        % 4. If you have a logo, use this includegraphics command to put it on the coversheet.
        \includegraphics[height=4cm]{OSU.jpg}   
        \par\vspace{.2in}
        \centering
        \scshape{
             Pavel Shonka, CS 461, Fall Term \par
            {\large\today}\par
            \vspace{.5in}
            \textbf{\Huge Problem Statement}\par
            \vspace{2in}
            
        }
        \begin{abstract}
        % 6. Fill in your abstract    
        	This project will solve the expense problems surrounding the tracking of smaller animals that can’t be fitted with a gps tracker. Currently they use planes flying overhead or someone hiking through the forest pointing a vhf antenna where they think the animals are. To solve this problem some sort of drone will be fitted with a vhf antenna and then flown over the area where the animals needing tracking reside. Outline in this paper is how we plan to make this work. Also outlined are a couple additions we hope to add to our finalized project. It is also explained how we will know that we have completed this project.
        \end{abstract}     
    \end{singlespace}
\end{titlepage}
\newpage
\pagenumbering{arabic}
\tableofcontents
% 7. uncomment this (if applicable). Consider adding a page break.
%\listoffigures
%\listoftables
\clearpage

% 8. now you write!
\section{Project Description}
For this project we will be designing and building a drone to carry a vhf receiver. Most animals are tracked using gps for ease and more data. Unfortunately, to track small animals like birds, skunks and smaller animals the large bulky equipment used for gps cannot be carried by those smaller animals. To make something light enough and use small amounts of power as to not weigh the animal down with a massive battery they use a vhf transmitter. These transmit vhf waves constantly while using very low amounts of power. Currently to track these animal’s researchers must go out with a handheld vhf receiver and get close enough to get a signal and then move around the animal so they can triangulate the signal to know an approximation of were the animal is. This takes a lot of time to find and track a single animal making current tracking of small animals very expensive. They can also track the animals by having someone in a plane pointing a vhf receiver out of the plane to find the animals. Even having to go through trees this method works well and a lot faster. We would like to make the method of tracking animals with vhf transmitters cheaper and less time consuming by using drones to take the place of airplanes.

\section{Solution}
Our proposed solution is to either use either a drone or a fixed wing drone to fly above the trees and track the animal. Using drone’s vs fixed wings has its advantages and disadvantages and we have yet to come to a decision on which to use. Drones are far more stable with the ability to hover and land in small spaces but cannot fly far or stay up very long. Carrying heavy devices like the tracking system could also use a lot of battery further restricting its flight distance. It is also more expensive as it would take far more electronic parts to fly. Fixed wing drones on the other hand can fly exceptionally far distances as well as carry heavy payloads efficiently. Fixed wings would also be cheaper, but we would need a clearing to land them. The hope is that the drone can come down and possibly recharge and then head back out without any human intervention. The drones will be able to triangulate the animals position while flying a path by taking several headings as to the signal direction as it flies in arcs across the survey area. As the drone flies over the area it will know its gps location and which direction it is heading. For the antenna setup we are thinking of doing a set of three with one point off to each side and one pointing straight ahead to help decipher what direction the signal is coming from. Based off that data the drone will either save the approximate location of the animal, so it can come back later to find its exact location or alter its course to determine the exact location of the animal before heading back to its planned flight path. As the Cs team our job will be to program and make the software to control the drone as well as use the signal data to find where the animal is or alter its path to better determine where the animal is. We will also need to use gps information to help control the drone and determine in which direction the antennas are pointing. Based off several headings for where the signal is coming from we will need to triangulate the animal and store that data for offloading to the researchers. We have talked about the ability to autonomously land the drone on a platform above the trees although depending on what type of drone and our time constraints we will try to implement this feature. In order to land the drone autonomously we will need to pinpoint the landing pad using video recognition or some other form of sensor data to line the drone up. Along with that the client hopes the drones can be charged automatically allowing for them to take of autonomously once fully charged.

\section{Performance Metrics}
To show that we have completed our project we will need to demonstrate our drone in the field. Our drone should be able to track more or as many animals in a short time period than someone on foot using a vhf receiver. We also need to be able to replicate the drone for under 6,000 dollars to sell to other research institutions or for our client to use a fleet to track all the animals in a determined area. Hopefully the ece team working on the drone will also complete the drone in the required time so we can have hardware to test on. After we have shown that it can track animals accurately and in a timely manner we will move on to trying to make it land autonomously as well as take off and charge without human help. If we can complete both stages, then we will have completed everything the client ever wished for if we can show him everything functions properly.

\end{document}