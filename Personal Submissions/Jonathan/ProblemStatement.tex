\documentclass[onecolumn, draftclsnofoot,10pt, compsoc]{IEEEtran}
\usepackage{graphicx}
\usepackage{amssymb}
\usepackage{amsmath}                                         
\usepackage{amsthm}
\usepackage{alltt}                                           
\usepackage{float}
\usepackage{color}
\usepackage{url}
\usepackage{balance}
%\usepackage[TABBOTCAP, tight]{subfigure}
\usepackage{enumitem}
%\usepackage{pstricks, pst-node}
\usepackage{setspace}
\include{pygments.tex}

\usepackage{geometry}
\geometry{textheight=9.5in, textwidth=7in}


% 1. Fill in these details
\def \CapstoneTeamName{         Taal Squad}
\def \CapstoneTeamNumber{		33}
\def \GroupMemberOne{			Jonathan Buntin}
\def \GroupMemberTwo{			Thomas Kuhn}
\def \GroupMemberThree{			Karl Popper}
\def \CapstoneProjectName{		Tracking radiotagged carnivores in the forest with automated drones and VHF antennas}
\def \CapstoneSponsorCompany{	Levi Lab, Department of Fisheries and Wildlife, Oregon State University}
\def \CapstoneSponsorPerson{		Taal Levi}

% 2. Uncomment the appropriate line below so that the document type works
\def \DocType{		Problem Statement
				%Requirements Document
				%Technology Review
				%Design Document
				%Progress Report
				}
			
\newcommand{\NameSigPair}[1]{\par
\makebox[2.75in][r]{#1} \hfil 	\makebox[3.25in]{\makebox[2.25in]{\hrulefill} \hfill		\makebox[.75in]{\hrulefill}}
\par\vspace{-12pt} \textit{\tiny\noindent
\makebox[2.75in]{} \hfil		\makebox[3.25in]{\makebox[2.25in][r]{Signature} \hfill	\makebox[.75in][r]{Date}}}}
% 3. If the document is not to be signed, uncomment the RENEWcommand below
\renewcommand{\NameSigPair}[1]{#1}

%%%%%%%%%%%%%%%%%%%%%%%%%%%%%%%%%%%%%%%
\begin{document}
\begin{titlepage}
    \pagenumbering{gobble}
    \begin{singlespace}
    	\includegraphics[height=4cm]{coe_v_spot1}
        \hfill 
        % 4. If you have a logo, use this includegraphics command to put it on the coversheet.
        %\includegraphics[height=4cm]{CompanyLogo}   
        \par\vspace{.2in}
        \centering
        \scshape{
            \huge CS Capstone \DocType \par
            {\large\today}\par
            \vspace{.5in}
            \textbf{\Huge\CapstoneProjectName}\par
            \vfill
            {\large Prepared for}\par
            \Huge \CapstoneSponsorCompany\par
            \vspace{5pt}
            {\Large\NameSigPair{\CapstoneSponsorPerson}\par}
            {\large Prepared by }\par
            %Group\CapstoneTeamNumber\par
            % 5. comment out the line below this one if you do not wish to name your team
            %\CapstoneTeamName\par 
            \vspace{5pt}
            {\Large
                \NameSigPair{\GroupMemberOne}\par
                %\NameSigPair{\GroupMemberTwo}\par
                %\NameSigPair{\GroupMemberThree}\par
            }
            \vspace{20pt}
        }
        \begin{abstract}
        % 6. Fill in your abstract    
            There has been very little research into small carnivores in the Pacific Northwest’s forests. Researchers have trouble tracking small carnivores due to their inability to carry bulky GPS devices. They use very high frequency (VHF) transmitters instead as they are more lightweight, however this requires constant work by field teams and much more time making it a much costlier approach. This projects goal is to deliver a physical device and the accompanying software to use an automated drone to track and triangulate the location of small VHF radio tagged carnivores. This will allow a wide variety of animals and insects to be studied much more in depth and efficiently.\par
        \end{abstract}     
    \end{singlespace}
\end{titlepage}
\newpage
\pagenumbering{arabic}
\tableofcontents
% 7. uncomment this (if applicable). Consider adding a page break.
%\listoffigures
%\listoftables
\clearpage

% 8. now you write!
\section{Problem Statement}

	There is very little research on many small carnivores in the pacific northwest. This is in large part due to the small animals inability to carry bulky and cumbersome GPS devices. Instead of using GPS to track and study these animals researchers use smaller and lighter very high frequency (VHF) transmitters. The problem with this method is that it requires constant work by field crews in difficult conditions to manually track the VHF signals and locate the animals. While this is both expensive and time consuming to accomplish, it also places people in hazardous situations and furthers the impact of the researchers on the animals being studied. If this process were automated, the use of VHF transmitters could potentially be adapted to track a wide variety of animals and insects including songbirds and butterflies.\par
    \vspace{5pt}
    The proposed solution is to automate this process and remove the need for field crews manually locating the animals by using drones instead. This would be done by fitting an aerial drone with a VHF receiver and programming it to fly above the forest along predetermined paths. The drone would be programmed to deviate from the flight paths to triangulate a precise location for the radio tagged animal. One of the drones would be tracking the locations of multiple animals at the same time. As the drone is triangulating one animal it can then use the signals from other animals within its range to help triangulate them as well. The system will need to be able to operate within a forest environment and record the data for analysis.\par
    \vspace{5pt}
    The project needs to produce a physical system to attach to the drone and the software to control the drone and locate the radio tagged animals. The software must be compatible with the chosen drone as it will need to have automated flight tracks and the ability to deviate from its flight paths. The software must also be compatible with the chosen VHF transmitters and receivers as the precise location of the radio tagged animal must be calculated and recorded for analysis. This system needs to reduce the time it takes to locate the animals in addition to reducing or removing the work done by field crews.\par


\end{document}
