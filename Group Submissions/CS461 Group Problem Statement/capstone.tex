\documentclass[onecolumn, draftclsnofoot,10pt, compsoc]{IEEEtran}
\usepackage{graphicx}
\usepackage{url}
\usepackage{setspace}

\usepackage{geometry}
\geometry{textheight=9.5in, textwidth=7in}

% 1. Fill in these details
\def \CapstoneTeamName{		TaalSquad}
\def \CapstoneTeamNumber{		33}
\def \GroupMemberOne{			Pavel Shonka}
\def \GroupMemberTwo{			Aidan Carson}
\def \GroupMemberThree{			Matthew Ruder}
\def \GroupMemberFour{			Camden Sladcik}
\def \GroupMemberFive{			Jonathan Buntin}
\def \GroupMemberSix{			Manuel Lara-Navarro}
\def \CapstoneProjectName{		Tracking radiotagged animals with automated drones and VHF antennas}
\def \CapstoneSponsorCompany{	Levi Lab, Department of Fisheries and Wildlife, Oregon State University}
\def \CapstoneSponsorPerson{		Taal Levi}

% 2. Uncomment the appropriate line below so that the document type works
\def \DocType{		Problem Statement
				%Requirements Document
				%Technology Review
				%Design Document
				%Progress Report
				}
			
\newcommand{\NameSigPair}[1]{\par
\makebox[2.75in][r]{#1} \hfil 	\makebox[3.25in]{\makebox[2.25in]{\hrulefill} \hfill		\makebox[.75in]{\hrulefill}}
\par\vspace{-12pt} \textit{\tiny\noindent
\makebox[2.75in]{} \hfil		\makebox[3.25in]{\makebox[2.25in][r]{Signature} \hfill	\makebox[.75in][r]{Date}}}}
% 3. If the document is not to be signed, uncomment the RENEWcommand below
%\renewcommand{\NameSigPair}[1]{#1}

%%%%%%%%%%%%%%%%%%%%%%%%%%%%%%%%%%%%%%%
\begin{document}
\begin{titlepage}
    \pagenumbering{gobble}
    \begin{singlespace}
    	\includegraphics[height=4cm]{coe_v_spot1}
        \hfill 
        % 4. If you have a logo, use this includegraphics command to put it on the coversheet.
        %\includegraphics[height=4cm]{CompanyLogo}   
        \par\vspace{.2in}
        \centering
        \scshape{
            \huge CS Capstone \DocType \par
            {\large\today}\par
            \vspace{.5in}
            \textbf{\Huge\CapstoneProjectName}\par
            \vfill
            {\large Prepared for}\par
            \Huge \CapstoneSponsorCompany\par
            \vspace{5pt}
            {\Large\NameSigPair{\CapstoneSponsorPerson}\par}
            {\large Prepared by }\par
            Group\CapstoneTeamNumber\par
            % 5. comment out the line below this one if you do not wish to name your team
            \CapstoneTeamName\par 
            \vspace{5pt}
            {\Large
                \NameSigPair{\GroupMemberOne}\par
                \NameSigPair{\GroupMemberTwo}\par
                \NameSigPair{\GroupMemberThree}\par
                \NameSigPair{\GroupMemberFour}\par
                \NameSigPair{\GroupMemberFive}\par
                \NameSigPair{\GroupMemberSix}\par
            }
            \vspace{20pt}
        }
        \begin{abstract}
        % 6. Fill in your abstract    
        	This document serves to define the core issue with current methods of animal tracking. The goal of this project is to reduce the cost of tracking forest animals and obtain reliable data. To address this issue, drones were declared a viable option if mounted with VHF antennas. With proper design and programming, the drones would be able to successfully locate tagged creatures, essentially reducing the amount of time, cost, and effort put into tracking these animals. In essence, drones shall be utilized to automate the tracking of these animals. In addition, altitude, wind, area coverage, and expenses shall be taken into consideration when designing these drones.
        \end{abstract}     
    \end{singlespace}
\end{titlepage}
\newpage
\pagenumbering{arabic}
\tableofcontents
% 7. uncomment this (if applicable). Consider adding a page break.
%\listoffigures
%\listoftables
\clearpage

% 8. now you write!
\section{Problem Description}
Tracking large animals is straightforward within open environments because of GPS tracking, which allows for pinpoint measurements from anywhere in the forest; however, the footprint of GPS transmitters along with the associated battery pack is too much for smaller animals, such as the Marbled Murrelet which the Department of Fisheries and Wildlife desperately wants to track. Current methods of tracking forest creatures involves utilizing VHF radio technology. Selected animals are given radio tags, which emit a unique radio frequency that is picked up by a VHF antennas. VHF antennas require manual assistance in order to successfully track creatures; as such, the cost of tracking using VHF antennas extremely expensive, manual, and inefficient. The ultimate goal of the project is to reduce the cost of tracking forest animals while obtaining reliable data. In order to accomplish this, several steps must be taken to construct a proper solution. The main objective is finding a VHF signal. The device designed must be able to receive VHF signals within densely forested areas. If the device cannot accomplish this task, then it does not solve the problem. Secondly, the device must be automated and capable of operating within a dense forest environment; as such, battery life must be taken into account. The constructed device must be able to last several hours in order to cover the largest area possible.

\section{Proposed Solution}
As stated within the problem section, tracking creatures within densely forested areas is costly within the Pacific Northwest. To fully automate the process of tracking creatures, drones equipped with VHF antennas will enable wildlife researchers indicators as to where tagged creatures are currently located. One of the proposed solutions involves designing a single drone equipped with a VHF antenna to patrol a designated route above the tree line. Researchers shall be able to assign a specific path for the drone with user friendly software. The biggest issue using this method is the battery of the drone must last a long time and charge quickly, otherwise it would be ineffective at searching or slower at searching than by hand respectively. To counteract this, fixed wing drones shall be used for increased flight speed along designated paths. With this style of flight, searching may return less accurate location information; therefore, the deployment of multiple drones may be implemented as to reduce flight path and increase the number of paths for greater accuracy. In addition, this would decrease the general flight time per drone and overall battery consumption. If autonomous drones followed specified search algorithms, then the location could be transmitted back to a central base camp, enabling researchers to locate animals in the shortest amount of time. As an additional feature, various sensor data on the drone, such as temperature, pressure, and humidity, can be recorded for each drone's area, essentially providing extra information of the regions of the forest.

\section{Performance Metrics}
The proposed completion of the project is when a working drone is delivered with capability of recording GPS coordinates and direction of strongest signal relative to itself. The project will have been indicated a success when there is a reduction in physical involvement needed for animal triangulation and/or a reduction in the monetary needs involved with the project. If money spent on manpower can be offset by a more reliable system that pays for itself over time then the project will have been indicated a success. Metrics for scoring a project are as follows:
\begin{itemize}
    \item physical manpower required
    \item total monetary spending over time
    \item quality of data captured (affected by data retrieval frequency, interval between sessions, etc.)
    \item reliability of system
\end{itemize}

If the solution can be used as a replacement to the current manual process because of a significant increase in one or more of these categories then the project will be considered a success. The system ideally should also be able to handle large areas of forest. This could as a baseline involve manually driving the drone to different parts of the forest, but if a long range drone or recharging stations could be built that allows for long range data retrieval, the manpower required for the project would be significantly reduced.

The end result of this project must result in a working prototype or multiple prototypes that can cover at least 100 square kilometers of radio tracking. The drones must be able to fly above the tree line of mountainous forested areas without crashing. The agreed upon cost of each drone should fall within the \$6,000 replication cost. The components and blueprints of each drone must be listed and printed such that the drones may be reproduced in the future. Programming and code of each drone must be available to the client with instructions on how to program the drones. The prototype should be able to collect data without manual intervention or assistance. The drones must locate animals faster than manual tracking. The data collected must accurately locate an animal within the forest. 
\end{document}