\documentclass[onecolumn, draftclsnofoot,10pt, compsoc]{IEEEtran}
\usepackage{graphicx}
\usepackage{url}
\usepackage{setspace}
\usepackage{cite}

\usepackage{geometry}
\geometry{textheight=9.5in, textwidth=7in}

\title{Tracking radio-tagged carnivores in the forest with automated drones and VHF antennas}
\author{Manuel Lara-Navarro}
\date{November 2018}

\begin{document}

\maketitle
\newpage


\section{Back-End: the technology implementing back end software}

    \subsection{PHP}
    An open source object-oriented language. Used by websites like facebook, wordpress, and flicker. Very similar to Python in regards to not needing compiling. Because of the similarity with
    python it can be a hard language to read if structure of code is not strict, known as “Spaghetti Code.” Very popular language because of vast community of developers, libraries, and
    frameworks. 80\% of modern websites use PHP today  \cite{ProgrammingLanguages}. Unlike Python and Java, PHP is really only suited for web-development purposes.

    \subsection{Python}
    An open source high-level aspect-oriented programming language. Very popular language among scientists. Python’s Django framework is used in popular apps Instagram, Pinterest, and
    EventBrite. Python was not intended to be used in web development. Less than 5\% of websites today use python \cite{ProgrammingLanguages}. Python is also considered a language that is easy to learn compared to
    languages like C++, Perl, and others. Python’s library support really makes it popular among all levels of programmers. Because of its well developed support python is used for almost all
    types of applications.

    \subsection{Java}
    An Object-oriented high-level class-based language programming language. Considered the most difficult to language to learn out of the three. The intricate syntax of Java compared to Python and PHP really separates it from the two. Java is also the only language out of the three that requires compiling. Java is clearly the most sophisticated language of the three. Java is referred to as a “write once and run everywhere” language being more universal and independent than Python and PHP. Java is the native language for the Android programming platform and must be considered when choosing platform implementation.
    \newline

    Because all of these languages have so much to offer, they would all be a great choice. The determining factor in choosing one over the other will depend on our platform.


\section{Hosting: the database technology used}

    \subsection{MySQL}
    Currently the most popular database for web-based applications. MySQL is a full feature
    open-source relational database management system owned by Oracle Corporation.
    Compatible with just about every Operating System. Used by Youtube, Facebook, Twitter
    and many other popular websites. MySQL has an abundance of resources and documentation
    to support it. It is extremely easy to learn how to use MySQL. Uses Structured Query
    Language to store and update data. Key terms used to manage the data are ‘SELECT’,
    ‘UPDATE’, ‘INSERT’, and ‘DELETE’. Structured by storing data into tables grouped into
    databases. MySQL has a ‘Strict’ schema with vertical scaling. MySQL can be best used
    for frequent updates and modifications of large volume of records, relatively small
    datasets, and data structure fits for tables and rows. Includes key features like
    Triggers, subSELECTs, and integrated replication support. Because of MySQL’s
    popularity, its ease of use, and its great support documentation it is a reliable
    option for database implementation.

    \subsection{MongoDB}
    A popular open-source document-oriented non-relational database. In MongoDB documents
    are created and stored in BSON files, Binary Javascript Object Notation format. Using
    JSON makes it easier to transfer data between servers and web apps and because of this
    MongoDB is the better option for storage capacity and speed due its greater efficiency
    and reliability. Unlike MySQL’s use of a ‘Strict’ schema, MongoDB uses a ‘Dynamic’
    schema eliminating the requirement for predefined structures of data. Dynamic schemas
    make MondoDB a much more flexible Database system than MySQL and very popular among
    businesses with rapid growth. MongoDB uses horizontal scaling which helps reduce the
    workload. Because of the structure of MongoDB, it can perform simple queries with
    high-performance compared to MySQL.

    \subsection{MariaDB}
    MariaDB is a relational database management system forked from MySQL by the original
    developers of MySQL with a stronger focus on being open-source. This system is
    compatible with just about every Operating System available. After MySQL was sold to
    Oracle, the community of open-source developers felt it was no longer open to the
    community enough for development and input. This led to the development of MariaDB.
    Today many large corporations and Linux distributions are using MariaDb like Google,
    Craigslist, and RedHat. MariaDB is still based off of MySQL and they have many
    similarities. The database structure and indexes are the same in the two. Having
    similar layout of data makes migration from MySQL to MariaDB a very easy task not
    requiring any modifications to data structure. MariaDB uses the same key terms as MySQL
    to store and update data, ‘SELECT’, ‘UPDATE’, and ‘INSERT’. MariaDB tends to be ahead
    of MySQL in terms of update releases and feature additions. It can be expected of
    MariaDB to be ahead in terms of features due to it being open-source and developed by
    the community. This can’t be said about MySQL due to it being maintained only by Oracle
    where they have a slower rate of updating and adding features. Currently MariaDB has
    higher performance rating than MySQL and is more open to the public.

\section{Power Source: the main source of power of the device}

    \subsection{Batteries}
    The simplest method would be to use removable batteries. Using batteries over fuel
    would eliminate the worry of spilling fuel on expensive material, having to depend on
    fuel, and hauling fuel around. Using batteries as the main source of power would also
    be the best way to recharge or refill the power source, because it could simply be
    swapped out for a secondary unit while the empty battery gets plugged in to recharge.
    Swapping and recharging batteries should be quicker than refilling fuel. The type of
    batteries used in drones today are Lithium material. One of the issues with lithium
    batteries is its tendency to burst into flames due to containing flammable liquid.
    Lithium batteries are the best option in terms of cycle life and capacity. New lithium
    battery technologies are constantly being developed. Current flight time with these
    batteries is about 20 to 30 minutes with quadcopters. New lithium batteries are
    expected to be released this year with double the flight time or half the size of
    current possibilities \cite{LiBatteries}. Using batteries for our drone will be the best realistic
    option and most affordable.

    \subsection{Hydrogen Fuel Cells}
    A pollution free form of power. The Conversion of hydrogen to electricity which leaves
    behind nothing but water and some heat. The advantages of hydrogen fuel cells are
    clearly environmentally friendly, but also much longer flight time and lifetime than
    traditional batteries. A disadvantage is however the cost. The upfront cost of
    hydrogen fuel cells is roughly 7 times that of current lithium batteries used in
    drones \cite{HydrogenFuelCells}. This is primarily due to the technology being a fairly unexplored one not yet
    mainstream especially in drone use. Another disadvantage is the need to find hydrogen
    fuel, obviously unlike batteries, hydrogen can’t be shipped. Eventually hydrogen fuel
    cells will be the main power source in the majority of drones, hopefully sooner than
    later.

    \subsection{Solar power}
    With the predicted use of solar panels expected to explode in the next few years, new technologies and solar cells are
    being developed. In the past, solar panels were bulky, stiff, and quite heavy. They were definitely not intended to
    be mounted on drones or really anything mobile. Today we can find smaller solar panels and even flexible ones that can
    easily be adjusted for fit. Currently there are many solar powered drones that can stay airborne for a much longer period of
    time than traditional battery powered drones \cite{SolarPower}. Today, a typical solar powered drone is not intended for long range flight or any
    payload. Solar power currently does not provide enough power delivery to fly for long distances in conditions such as those
    required by our drone where we need to carry a bit of weight to perform our readings. In the near future solar power will
    be a dependent main source of power for drones but today it would be more realistic as a source of backup power.


\section{Conclusion}

    After spending time researching these technologies we can compare how they can be used in this project.
    The Back-End could be implemented using any of the technologies discussed. All three are great options,
    in the end the choice will come down to what exactly the platform used in this project requires and how it
    must be designed. The hosting and database technology will work either of the pieces discussed, but the best
    choice would be the the traditional MySQL due to its popularity and support. The choice for power source will
    be the only one that is a realistic option today. Batteries are the safest and most reliable option for this
    project.

\newpage
\bibliography{bib}
\bibliographystyle{IEEEtran}


\end{document}
