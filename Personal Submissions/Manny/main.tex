\documentclass[10pt, draftclsnofoot,onecolumn, letterpaper]{IEEEtran}
\usepackage[utf8]{inputenc}
\renewcommand{\familydefault}{\sfdefault}

\title{Problem Statement}

\author{
  Manuel Lara Navarro\\
  \and
  Senior Software Engineering Project\\
  \and
  Fall 2018\\
  \and
  October 11 2018\\
}

\begin{document}

\maketitle

\section{Project abstract summarizing the entire document}

The purpose of this project is to aid the Department of Fisheries and Wildlife to track study smaller animals. Currently there is a lack of studies on pacific northwest small carnivores. This is large in part due to lack of technology. Without modern technology the process of being able to study these smaller animals requires physical labor in semi-risky environments. With the help of modern technologies, this unstudied category of pacific northwest animals will soon be measured. This document will explain in further detail what the current situations and issues are for trying to study these smaller carnivores. We will also look at what ideas have been brainstormed and how we might develop or add them to what is available. Finally, we will assess what we think we can accomplish with this task and what we expect to get out it in performance and functionality.

\newpage{}

\section{Definition and description of the problem you are trying to solve written for a general, but educated audience}

The department of fisheries and wildlife would like to begin studying small carnivores in the Northwest. This group of small carnivores include ermine, long-tailed weasels, and civet cats (spotted skunks). These animals have been hard to track because they are too small to carry Global Positioning System(GPS) and battery equipment. With new technologies such Very High Frequency (VHF) transmitters which are much smaller and lighter. Currently, the method that is used to find these tagged animals requires field crews to be on the ground physically tracking them with antennas. This method consumes too much time and tends to be in difficult conditions, sometimes even at night. Obviously, this could put individuals in dangerous situations with high risk. Because of this treacherous task a lot of training is also necessary which further adds to countless hours and physical labor.

\section{Proposed solution}

We will modify or develop a drone that will be able to use software simulating a vhf antenna tracking system. We can do this by either attaching an antenna with its proper technology or making it part of our drone which gives us the same functionality built in. If we choose to attach the antenna, we would need to make the drone work and be fully compatible with it. If we choose to build a drone with its own antenna, we would need to create the software from basically the ground up. We will take what the client has envisioned or planned, and we will find what is possible and realistic. We will design and build a prototype of that blueprint with a team of ECE and CS students. The device will consist of technologies such as The Global Positioning System and software that integrates or simulates VHF signal tracking. The drone will use the signal tracking software to pick up and read radio waves being transmitted from the animals that have been previously radio tagged. Using the signal readings, the drone will use triangulation to calculate and pinpoint the precise location of the animal being tracked. The drone will either store all this data internally or relay it back instantly to a host. The information attained by the drone will be formatted and produced to be readable by the crews on the ground, at the launch site or even back in the office or lab.

\section{Performance metrics: Tell how you will know when you have completed the project. Metrics help you and your client agree on what successful completion of the project will look like}

We will consider our project complete when we have a functional prototype that the client is satisfied with. That is when a prototype can functionally perform the desired tasks. We will expect the device to be ready to produce results in real life tests. The biggest objective of this project will not only be to develop and produce the device itself but to also create a technology that is usable. The device must be one that can be learned and used functionally in a relatively short amount of time. In order for this to be achieved we must remember to not reinvent the wheel. We will aim to take existing technology and implement into ours to make it extremely universal. Once we can say our technology is simple enough to be taught to the user in less than a day, we will consider it ready. We will also need to supply a “Write up” or manual for the device. The manual can either be in a form of a write up or instructional documentation. In this document we will explain technological details such as expected battery life and more. We will expect this documentation to fully supply a new user with the all the information they need about the device and even a “How-To” on using and learning the device and its technologies. Once we meet these objectives we will deliver a complete project as a prototype with necessary functionalities and documentation.

\end{document}
